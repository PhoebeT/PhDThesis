\chapter{Introduction}
\label{introchap}

This thesis makes the bold attempt to utilize a new and emerging technology: coherent extreme ultraviolet light generated via high harmonic generation to study the fundamental science behind magnetism. Coherent light at short wavelengths (and thus high photon energies) has proven to be extremely useful for a variety of applications and fundamental science. In this thesis, coherent light that is resonant with ferromagnetic elements is used to probe dynamics that occur at extremely short (femtosecond) timescales.

In Chapter 1 we briefly outline the motivation for this work, both fundamental: based on the current state of understanding in ferromagnetism, and technical: developing a path towards next-generation magnetic storage devices. Chapter 2 provides a first-order theoretical description of the modern day understanding of ferromagnetism: beginning from the an atomic picture of the He atom, and proceeding to the simplest molecular description. Here we will describe the two most commonly used Hamiltonians in solid state physics: the Heisenberg and Hubbard Hamiltonians, and discuss which specific situations warrant the use of either picture. We also provide an introduction into the difficultly of solving the complicated many-body systems studied in this thesis. In Chapter 3 we provide a theoretical background for the light-matter interaction in the context of magnetism in solids. In chapter 4 we describe the experimental apparatus used for most of the work in this thesis, along with how we generate the very special extreme ultraviolet (EUV) light that is central to this work. We show how characterization techniques based on high harmonic generation (HHG) have the unique opportunity to uncover new insights into the physics of these challenging systems. Next, in Chapters 5-7 we describe several of the experiments performed with our setup and the insights into fundamental scientific questions uncovered. We conclude with an outlook on some future experiments to be continued by the current and future members of the magnetism team in the KM Group.


