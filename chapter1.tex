\chapter{Introduction}
\label{introchap}

This thesis makes the bold attempt to utilize a new and emerging technology: coherent extreme ultraviolet light generated via high harmonic generation to study the fundamental science behind magnetism in transition metals and their compounds. Coherent light at short wavelengths (and thus high photon energies) has proven to be extremely useful for both fundamental science and a variety of applications. In this thesis, coherent light that is resonant with ferromagnetic elements is used to probe dynamics that occur at extremely short (femtosecond) timescales. In this way, we are able to detect the microscopic processes that allow for the first demonstration of direct optical manipulation of spins.

In this chapter I briefly outline the background and motivation for this work: the history of the field of ultrafast ferromagnetism and the forces that continue to drive research on ultrafast magnetism. This motivation is fundamental: based on the current state of understanding in ferromagnetism, and also technology driven: developing a path towards next-generation magnetic storage devices. Chapter 2 provides a first-order theoretical description of the modern day understanding of ferromagnetism: beginning from an atomic picture of the He atom, and proceeding to the simplest molecular description. Here I will describe the two most commonly used hamiltonians in solid state physics: the Heisenberg and Hubbard hamiltonians, and discuss which specific situations warrant the use of either picture. The study of these models provides an introduction to the difficultly of predicting the behavior of the complicated many-body systems studied in this thesis. In chapter 3 I derive a theoretical background for the light-matter interaction between extreme ultraviolet and soft x-ray light and magnets. I describe the experimental apparatus used for most of the work of this thesis, along with how we generate the very special extreme ultraviolet (EUV) light that is central to this work. I show how characterization techniques based on high harmonic generation (HHG) have the unique opportunity to provide new insights into the physics of these challenging systems. Next, in chapters 4-6 I describe several of the experiments performed with our setup and the insights into fundamental scientific questions that were uncovered. Chapter 4 introduces experimental measurements taken on single crystal nickel (111), using both magneto-optical spectroscopy and angle-resolved photoemission. These measurements were the first to observe the spectroscopic signatures of an extremely fast and short-lived spin excitation in the ultrafast response to a driving infrared laser pulse. In the next chapter, I demonstrate how the existence of this ultrafast spin excitation allows us to map out all of the future dynamics of the system using only the knowledge of the intensity of the driving laser pulse. In the final experimental chapter, I present direct experimental evidence for optically driven spin transfer between elemental sublattices within the time duration of the driving laser pulse. This work is the first to demonstrate direct optical manipulation of spins. I conclude with an outlook on some future experiments to be continued on these topics using the experimental apparatus described in this work.

\section{Fundamental motivation and results}

Magnetization in magnetic materials can be strongly suppressed by ultrafast laser irradiation on femtosecond timescales\cite{E.BeaupaireJ-CMerleA.Daunois1996}. The first experiment to show that magnets could be transiently demagnetized by a femtosecond laser pulse were done in 1996. This result was surprising at the time because previous to that experimental discovery, people believed that dynamics in ferromagnets could never proceed on timescales faster than hundreds of picoseconds, due to their large exchange energy. Following that seminal discovery, many experiments have been performed on transition metal ferromagnets (Co, Ni, and Fe) to show that the magnetization is quenched within $\approx$100 to 500 fs, before subsequently recovering within tens of picoseconds \cite{ Koopmans2005, Rhie2003,  Stamm2007}. Eventually, with the help of extensive atomistic and density functional theory models, the question instead became: why is the demagnetization process so slow?

This dynamic process has been studied using many experimental techniques, including magneto-optical spectroscopy \cite{Carpene2015,Koopmans2005,La-O-Vorakiat2009,Mathias2012,Roth2012,Turgut2016}, photoelectron spectroscopy \cite{Carley2012,Eich2017,Rhie2003}, and x-ray magnetic circular dichroism \cite{Stamm2007,Boeglin2010}. In the KM group, magnetic materials were previously studied with extreme ultraviolet (EUV) transverse magneto-optical Kerr effect (TMOKE) measurements \cite{La-O-Vorakiat2009} and angle resolved photoemission (ARPES) \cite{Chen2017}, however this thesis is the first to combine the two spectroscopic techniques in order to understand exactly what each spectroscopy was measuring, and to show that when driven with the right fluence, the material can actually undergo a true magnetic phase transition \cite{Tengdin2018}. 

This insight was critical because the underlying physical mechanisms that drive ultrafast magnetization dynamics are still under debate. A number of microscopic models based on mechanisms such as Elliott-Yafet spin-flip scattering \cite{Koopmans2005,Koopmans2010}, dynamic exchange splitting reduction \cite{Mueller2011,Mueller2013,Krauß2009}, as well as ultrafast spin-polarized or unpolarized currents \cite{Battiato2010,Eschenlohr2013} have been proposed. In addition, coherent optical excitation \cite{Bigot2009}, spin-orbit coupling \cite{Tows2015,Zhang2000}, and collective magnon excitation \cite{Carpene2015,Zhang2012,Schmidt2010} are also believed to play an important role in this process. In the past, the difficulty in determining the correct underlying mechanism was due to several issues: first, standard magneto-optic spectroscopies are simply not sensitive to highly nonequilibrium excited magnetic states; they cannot simultaneously monitor the coupled electron, spin, and lattice degrees of freedom (d.o.f.). Second, these spectroscopies average over different depths of the material, which masked the physics of the ultrafast phase transition. 

In this thesis, we experimentally investigate nickel (111) using two advanced spectroscopy techniques based on extreme ultraviolet light from high harmonic generation. Combining these techniques allowed us to discover that the ultrafast magnetic phase transition in nickel occurs significantly faster (within 20 fs of laser excitation) than it was previously thought to occur \cite{Tengdin2018}.

\section{Application driven motivation and results}

This thesis also includes the first ultrafast element resolved measurement of Co$_2$MnGe, a complex half-metallic ferromagnet. This material proved have extremely interesting behavior that can only be accessed by a resonant, element-selective probe with very high time resolution such as EUV Tr-TMOKE. The material is important because its special bandstructure allows it to have unique and technologically important properties such as the ability to conduct a single spin state.

Ultralow-power, high-performance nonvolatile memory and logic devices based on magnetic spin (“spintronics”) are starting to make inroads into conventional computing and represent prime candidates for practical quantum technologies. However, fully exploiting the capabilities of new materials and technologies will require a detailed understanding of the underlying physics, as manifested by nanoscale dynamic magnetization properties. At present, our understanding of spin interactions is crude and predominantly phenomenological: a comprehensive, self-consistent, microscopic model that rigorously includes the spin, electronic, photonic and phonon-degrees of freedom and their interactions does not yet exist. This understanding is fundamentally constrained in large part by a limited ability to directly observe magnetism on all relevant time and length scales. While the fundamental length- and time-scales for magnetic phenomena are nanometers (exchange length) and femtoseconds (exchange splitting), tools that enable the exploration of dynamics at these scales have only recently become available. 

Band structure engineering can significantly enhance the capabilities of quantum materials by controlling their magnetic and transport properties \cite{He2017,Klaer2009}. One particular focus is the development of materials with the ability to efficiently generate and sustain pure spin currents for use in spintronics devices. Half-metallic Heusler compounds are particularly promising candidates for this application due to their unique band structure \cite{Muller2009, Mann2012}: one spin-channel (the majority band) is metallic in nature, but the other spin-channel (minority) is insulating, with a band-gap at the Fermi energy (see Fig. 1). This property is of intense interest for spintronics applications because it enables spin-based logic devices such as transistors, diodes, and gates \cite{Prinz1995}. It also puts unique restrictions on the spin excitation pathways that can be driven by a coherent laser pulse in each element of the compound, which can lead to exciting new dynamics, such as manipulating the magnetic state of elements in a compound (\cite{Elliott2016,Steil2010}). Although all-optical switching has been demonstrated experimentally in both ferrimagnetic and ferromagnetic materials using circularly and linearly polarized light, the response of the material is not instantaneous, but takes place on picosecond timescales – long after the laser excitation pulse \cite{Kuiper2014}, and in some cases also requires the cumulative effect of many pulses \cite{Lambert2014,Mangin2014}.

To date, researchers have investigated several Heusler alloys using ultrafast femtosecond laser pulses to drive the system into a nonequilibrium state while monitoring the response of the system with magneto-optics \cite{Muller2009, Mann2012,Steil2010,Wustenberg2011}. These experiments can help to reveal the rate at which the total magnetization dynamics evolve in a material and estimate how different microscopic mechanisms contribute, driving further development of functional materials. However, visible lasers probe the net magnetization averaged over all elements in the material. Very recent theoretical papers exploring laser-excited Heusler compounds have suggested the possibility of light-induced spin transfer from one element to another on extremely fast ($<$10 fs) timescales \cite{Dewhurst2018,Elliott2016}. This could in theory enable the ultimate goal of ultrafast direct optical manipulation of the magnetic state of a material, provided these dynamics can be observed.

Ultrafast extreme ultraviolet (EUV) high harmonic pulses make it possible to uncover the element-specific spin dynamics in multi-component magnetic systems, providing rich new information not accessible using visible light. Recent work has explored ultrafast laser-induced spin dynamics in ferromagnetic alloys and multilayers, where distinct responses –– such as the existence of a time lag between the quenching of the magnetization of different elements in an FeNi alloy –– were observed in the KM Group previously \cite{Mathias2012}. In the sixth chapter of this thesis we provide results demonstrating that a single ultrafast laser pulse can directly transfer spin polarized electrons, and thus magnetization, from one magnetic sublattice to another within the time duration of the excitation pulse. This measurement was made possible by the element specific nature of the EUV probe used to map out the dynamics of each element's magnetic state in time. The magnetization of cobalt is transiently enhanced during the laser pump pulse, while that of manganese quenches rapidly. This extremely fast increase in magnetization observed in the cobalt sublattice is the fastest change in magnetization observed to date and a unique property that stems from the half-metallic character of the material. The possibilities for exploiting the unique behavior of this material extend to the realm of switches and triggers that can operate on few femtosecond or even attosecond timescales.