\chapter{Concluding Remarks and Future Directions}

In the first two chapters of this thesis, I briefly described the physics of ferromagnetic materials from an atomic, molecular, and crystalline lattice picture. I showed that the solutions to the equations for a many-body system such as the one relevant for magnetic effects quickly becomes too complicated to solve analytically. Instead, we rely on approximations that can give us a qualitative understanding and also a quantitative one, such as is the case for density functional theory. Next, I outlined the origin for resonant magneto-optical effects by deriving the transition matrix element for resonant interaction between X-ray radiation and matter in an atomic picture. I showed that the true transition matrix elements for a solid can be constructed from combinations of the atomic ones. In this way, I show that using magneto-optics, we can directly probe both the charge and spin dynamics using light resonant at the 2p or 3p edges, which in our case is generated from high harmonic generation.

Next I outline several experiments that represent a significant step towards the ultimate goal of direct manipulation of ferromagnetic spins with single pulses of light. In chapters 4-5, we show that the ultrafast demagnetization process discovered over 20 years ago is actually driven by a nonequilibrium spin distribution that occurs within the first 20 fs after excitation by a femtosecond laser pulse, an order of magnitude faster than previously suspected. After this surprising discovery, we studied spin dynamics in another system, one that proved to have a favorable band structure for the direct manipulation of spins via light. We showed that within the duration of the driving laser pulse, the magnetic moment of one element is transiently enhanced, while that of other is decreased. The observation was made possible by the element specificity of the tools used to study the dynamics of the material. Using extreme ultraviolet light resonant with the shallow core levels of Co and Mn allows us to simultaneously and independently track the dynamics of the two elements as the material responds to ultrafast laser excitation. In a separate simulation based on the classical Landau-Lifshitz Gilbert equations, we showed that these surprising short time dynamics cannot be captured in a LLG simulaton, and thus must be due to quantum mechanical effects. These results are presented in Appendix B. In fact, the LLG equation predicts that the Co decay should precede that of the Mn, which is the opposite of what we observed experimentally.

In the future, the magnetics and ARPES teams in the KM group will attempt to undertake a number of important further experiments based on the results described here. First, a series of experiments are planned with single crystal iron and cobalt, to experimentally test whether the spin excitation observed in nickel can be generalized to other simple ferromagnets. The Curie temperature in Cobalt is the highest, and so the measurements will first investigate Fe and then Co, in hopes that the electrons can be driven above the Curie temperature, and the true ferromagnetic-paramagnetic phase transition induced. We also plan to carry out experiments using a longer wavelength pump, such as 1.3 $\mu$m, in order to drive the phase transition without burning the sample.

Separately, future experiments on the Co$_2$MnGe will be carried out by the magnetics team to investigate several outstanding questions arising from the new results outlined in this thesis. First, we will attempt to shorten the driving laser pump pulse using a plate pulse compression technique that has been previously used successfully for low power femtosecond pulses \cite{Cheng2016}. We will shorten the pump to 10 fs, to test whether the transient enhancement of Co is correspondingly shortened. Additionally, the transition probabilities based on the DFT calculated band structure and driving laser energy allow us to make some predictions about the behavior of the material when excited by laser pump pulses with different photon energies. We will investigate these predictions by pumping the sample with 3 eV photons generated via frequency doubling in a BBO crystal.

There are also other interesting half-metallic heusler systems that can be studied using the experimental apparatus outlined in this thesis. Particularly, Ni$_2$MnSb is a promising candidate, for which predictions have been made that the enhancement could be even stronger than in Co$_2$MnGe \cite{Elliott2016}. We also have plans to acquire single crystal Co$_2$MnGa so that it can be studied using both Tr-TMOKE and ARPES.

Finally, we conclude by remarking that the results presented in this thesis would never have been possible without an enormous number of contributions from previous and current members of the KM Group in Boulder. The technical contricutions made in this dissertation were primarily to improve the data acquisition and stability of the experimental apparatus so that acquiring the results presented here became possible. It is important to note that the experimental magnetic asymmetry signal present in the heusler half-metal system investigated in chapter 6 is only 10 percent that of the signal available in bulk conventional ferromagnetic materials, and even only 5 percent of the signal possible from optimally chosen ferromagnetic thin films. Technical improvements made to the setup such as active beam stabilization, chirped mirrors, and better data aquisition methods, were able to improve long term stability and brightness of the high harmonic source so that statistics about the dynamics could be acquired over the course of multiple days, with fully repeatable and sufficiently low-noise results.

These improvements to the experimental apparatus make additional experiments possible, and improve the usability of the setup. Following this first demonstration of the EUV MOKE method on complex compound half-metal material Co$_2$MnGe, there are many more complex materials that can be studied using this same experimental apparatus. There are truly no limits in sight!