\chapter{Concluding Remarks and Future Directions}

In this thesis we outline several experiments that represent a small step towards the ultimate goal of direct manipulation of spin with pulses of light. In chapters 4-5, we show that the ultrafast demagnetization process that was discovered over 20 years ago, is actually driven by a nonequilibrium spin distribution that occurs within the first 20 fs after excitation by a femtosecond laser pulse, an order of magnitude faster than previously suspected. After this surprising discovery, we moved towards studying spin dynamics in another system, one that proved to have a favorable band structure for the direct manipulation of spins via light. We show that within the duration of the driving laser pulse, the magnetic moment of cobalt is transiently enhanced, while that of manganese is decreased. The observation is made possible by the element specificity of the tools used to study the dynamics of the material. Using extreme ultraviolet light resonant with the shallow core levels of Co and Mn, we are able to simultaneously and independently track the dynamics of the two elements as the material responds to ultrafast laser excitation. In a separate simulation based on the classical Landau-Lifshitz Gilbert, we show that these short time dynamics cannot be captured in a simulaton that does not include quantum mechanical effects. In fact, the LLG equation predicts that the Mn decay precedes that of the Co, which is the opposite of what is observed experimentally.

In the first two chapters of this thesis, we briefly describe the physics of ferromagnetic materials from an atomic, molecular, and crystalline lattice picture. We show that the solutions to the equations for a many-body system such as the one relevant for magnetic effects quickly becomes too complicated to solve analytically. Instead, we rely on approximations that can give us a qualitative understanding and also a quantitative one, such as is the case for density functional theory. Next, we outline the origin for resonant magneto-optical effects by deriving the transition matrix element for resonant interaction between X-ray radiation and matter in an atomic picture. We show that the true transition matrix elements can be constructed from combinations of the atomic ones. In this way, we show that using magneto-optics, we can directly probe both the charge and spin dynamics using light resonant at the 2p or 3p edges, which in our case is generated from high harmonic generation.

In the future, the magnetics and ARPES teams in the KM group will attempt to undertake a number of important further experiments based on the topics studied here. First, a series of experiments are planned with single crystal iron and cobalt, to experimentally test whether the spin excitation observed in nickel can be generalized to other simple ferromagnets. The Curie temperature in Cobalt is the highest, and so the measurements will first investigate Fe and then Co, in hopes that the electrons can be driven above the Curie temperature, and the true ferromagnetic-paramagnetic phase transition induced.

Separately, future experiments on the Co$_2$MnGe will be carried out by the magnetics team to investigate several outstanding questions arising from the new results outlined in this thesis. First, we will attempt to shorted the driving laser pump pulse using a plate pulse compression technique that has been previously used successfully for low power femtosecond pulses. We will to shorten the pump to 10 fs, to test whether the transient enhancement of Co is correspondingly shortened. Additionally, the transition probabilities based on the DFT calculated band structure and driving laser energy allows us to make some predictions about the behavior of the material when excited by laser pump pulses with different photon energies. We will investigate these predictions by pumping the sample with 3 eV photons generated via frequency doubling in a BBO crystal.

There are also other interesting half-metallic heusler systems that can be studied using the experimental apparatus outlined in this thesis. Particularly, Ni$_2$MnSb is a promising candidate, for which predictions have been made that the enhancement could be even stronger than in Co$_2$MnGe \cite{Elliott2016}.

Finally, we conclude by remarking that the results presented in this thesis would never have been possible without an enormous number of contributions from previous members of the KM Group. The technical step taken forward in this dissertation has primarily been to improve the data aquisition and stability of the experimental apparatus so that acquiring the results presented here became possible. It is important to note that the experimental magnetic asymmetry signal present in the heusler half-metal system investigated in chapter 6 is only 10 percent that of the signal available in bulk conventional ferromagnetic materials, and even only 5 percent of the signal possible from optimally chosen ferromagnetic thin films. Technical improvements made to the setup such as active beam stabilization, chirped mirrors, and better data aquisition methods, were able to improve long term stability and brightness of the high harmonic source so that statistics could be aquired over the course of multiple days, with fully repeatable and sufficiently clear results.